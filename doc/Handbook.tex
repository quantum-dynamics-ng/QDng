\documentclass[a4paper,12pt]{scrbook}
\usepackage[utf8]{inputenc}
\usepackage{environ}
\usepackage{amsmath}
\usepackage{hyperref}

%% Macros
\newcommand{\option}[2]{\item \texttt{#1}\\ #2}
\newcommand{\code}[1]{\texttt{#1}}

\NewEnviron{options}{\subsubsection*{Options}{\begin{itemize}\BODY\end{itemize}}}


\NewEnviron{example}{\subsubsection*{Example}
\begin{verbatim}
  \BODY
\end{verbatim}
}


\title{QDng Docomentation}
\author{Markus Kowalewski}
\begin{document}
\maketitle

\tableofcontents

\chapter{Input Format}
\section{Input files}
The native QDng input format is XML but there is a simplified version which allows
for easier typing:
\begin{verbatim}
 qndg { attr1=value1 attr2="More_Complex/Value"
   prog1 {
     ...
   }
}
\end{verbatim}
which is equivalent to 
\begin{verbatim}
<qdng attr1="value1" attr2="More_Complex/Value"
   <prog1>
     ...
   </prog1>
</qdng>
\end{verbatim}
Values containing only letters, numbers, and the characters \verb|.-_:,| need not to
be quoted in the simplified format.

\section{Operator References}
To save memory and initialization time it is possible to define access keys to operators
so that they can be reused at another place in the programm. By adding the option \verb|key|
to an operator it is given a reference. Instead of re-defining the operator in another place
it can be re-used with \verb|ref|.

For example in a multistate system the kinetic hamiltonian
is only needed once:
\begin{verbatim}
hamiltonian { name=Multistate
   m0.0 { name=Sum
      Tkin {name=GridNablaSq dims=1 mass0=1763.2 key=T}
      Vpot {name=GridPotential file="potX"}
   }
   m1.1 { name=Sum
      Tkin {ref=T}
      Vpot {name=GridPotential file=potA}
   }
} 
\end{verbatim}

Operators can also be defined globaly in an extra \verb|opdefs| section to make them
available in several subprograms. The entries are called \verb|opdef| for normal operators
and \verb|udef| for propagators:
\begin{verbatim}
qdng {
   opdefs {
      opdef {name=GridNablaSq dims=1 mass0=1763.2 key=T}
      opdef {name=GridPotential file="../../tests/GridPot0" key=V}
   }
   eigen {dt="5" steps="1000" Nef="10" dir="TGE-GSPO-im" method="imag"
           propagator {name="GSPO"
                 A {ref=T}
                 B {ref=V}
           }
           wf {file="../../tests/GridGauss" normalize="true"}
   }
   eigen {dt="5" steps="2048" dir="TGE-GSPO-ac" method="ac"
           propagator {name="GSPO"
                 A {ref=T}
                 B {ref=V}
           }
           wf {name="LC" normalize="true"
              wf0 {file="TGE-GSPO-im/ef_0"}
              wf1 {file="TGE-GSPO-im/ef_1"}
              wf2 {file="TGE-GSPO-im/ef_2"}
              wf3 {file="TGE-GSPO-im/ef_3"}
              wf4 {file="TGE-GSPO-im/ef_4"}
              wf5 {file="TGE-GSPO-im/ef_5"}
           }
   }
}
\end{verbatim}

\section{Variables and Loops}
TODO
\begin{verbatim}
for $loopvar = <begin> <end> [step]
{
....
}
\end{verbatim}

\begin{verbatim}
$a = "1234"
$(a)
$(a) %.2f
\end{verbatim}

\begin{verbatim}
for $delay = 1 5 {
   propa { dt="2" steps="2000" wcycle="2000" dir="propa" ~ $(delay) 
      propagator { name="Cheby"
         hamiltonian { name="Sum"
            A { name="HermitianMatrix" file="LevelH2mat"}
            B { name="MatDipole" file="LevelDIPmat" laser= "Laser_" ~ $(delay) %.2f }
         }
      }
      wf { name="Level" file="Level_ef_0"}
   }
} 
\end{verbatim}


\chapter{Data Format}

\section{Raw Data with Additional Meta-Data}
raw data file + .meta file in text format.
\subsection{Wave functions}

\subsection{Operators}

\section{Protobuf format}
Single file containing data and meta data.
MATLAB interface for creation and reading provided.


\chapter{Main Programms}

\begin{verbatim}
qdng [ -d output directory] [-p cpus] <input file> [var1=val1] ... [varN=valN] 
\end{verbatim}

\subsection{Global Options}
Global options can be given in several locations:
\begin{itemize}
 \item directly in the input file.
 \item in ~/.qdconfig (as key-value file).
 \item in .qdconfig in the current directory (as key-value file).
\end{itemize}

\begin{options}
 \option{debug}{Print more information.}
 \option{modpath}{Search path for dynamic loadable modules.}
\end{options} 

\subsubsection{Buffers}
\begin{options}
 \option{MaxBufferSize}{Buffer size for wave function buffers. Used for example in SIL and OCT. The value is in bytes but
 can suffixed by K,M or G.}
 \option{tmpdir}{Path for on-disk buffer files.}
\end{options}

\subsubsection{Wave Function File Format}
\begin{options}
 \option{wf\_format}{Wave function output format: binary or stream  [stream]}
 \option{compress}{Turn on/off wave function compression [false].}
 \option{comptol}{Tolerance value for seting values to zero.}
 \option{complevel}{Compression level like defined in zlib/bzip2 (1=fast, 9=best but slow).}
 \option{compmethod}{Method ot use. Chose ZLIB or BZIP [ZLIB].}
\end{options}

\subsubsection{FFTW}
\begin{options}
 \option{wisdom}{File name for reading wisdom [wisdom].}
 \option{wisdomsave}{File name for saving wisdom [wisdom].}
 \option{fftwOptimize}{Try to optimize fft strategy. This can take some time but accelerate large ffts.}
\end{options}


\section{Propagation}
Wave function propagation.
\subsubsection*{Standard}
\begin{options}
\option{dt}{time step. A negative time step means backward propagagation. (required)}
\option{steps}{number of time steps (required)}
\option{wcycle}{number of step to make until a wf will be written.}
\option{fname}{basename for the propagated files [WF].}
\option{start}{Index for the first written wave function (default: 0). if $>0$ the initial wave function is not written.}
\option{dir}{Output directory.}
\end{options}

\subsubsection*{Printing}
\begin{options}
\option{norm}{Print norm in the report [true].}
\option{nfile}{Filename -- write norm to seperate file. If not given the norm is printed to stdout.}
\option{energy}{Print the Energy in the report [true].}
\option{proj0}{Print the projection against in the initial wave function [false].}
\option{proj0Abs}{Print the absolute square of the projection against in the initial wave function [false].}
\option{spectrum}{Filename -- if given a the projection against the inital wavefunction is turned on, the power spectrum of the propagagation will be written.}
\end{options}

A \verb|propagator| and \verb|wf| section is needed to define the problem. Optional the initial condition can
be modified with in the \verb|filterinit| section. For analyzing or modifying the results before and after each time step
the \verb|filterpre| and \verb|filterpost| sections can be used. For details see sec. \ref{sec:Filters}.

\subsubsection*{Examples}
A simple propagation of super position in a 2D system:
\begin{verbatim}
<propa dt="5" steps="10" wcycle="10>
   <propagator name="SPO">
         <Tkin name="GridNablaSq" mass="1.5979,1.0645"/>
         <Vpot name="GridPotential" file="pot_file"/>
   </propagator>
   <wf name="LC">
        <wf name="GridCartesian" file="ef_0"/>
        <wf name="GridCartesian" file="ef_1"/>
   </wf>
</propa> 
\end{verbatim}

Propagation of a 1D System. The eigenfunction is given a momentum of -5 au before starting the
propagation. The position and momentum expetation values are evaluated after every time step:
\begin{verbatim}
<propa dt="16" steps="4096" wcycle="1000" dir="propa_kick"
 nfile="norm" spectrum="autocorr">
        <propagator name="GSPO">
              <A name="GridNablaSq" mass="1763.2"/>
              <B name="GridPotential" file="potX"/>[false|true] Integriert über den Wert über die Zeitschritte auf.
        </propagator>
        <wf name="LC" normalize="true">
                <wf file="ef_0"/>
        </wf>
        <filterinit>
                <apply name="GridKick" k="-5"/>
        </filterinit>
        <filterpost>
                <expeconly name="GridMomentum"/>
                <expeconly name="GridPosition"/>
        </filterpost>
</propa> 
\end{verbatim}


\section{Eigenvalues}
Find the Eigenvalues to given operator.

\begin{options}
\option{method}{Method to use: \code{imag} imaginary time propagation, \code{ac} project efs from autocorrelatin, \code{lanczos} obtain efs by lanczos iterations.}
\option{dt}{Time step.}
\option{dir}{Output directory.}
\option{fname}{Basename for eigenfunction files [ef].}
\option{ename}{Alternative name for energy.dat}
\option{diag}{diagonalize hamiltonian in ef basis [true for method imag].}
\end{options}

\subsubsection*{Imaginary time propagation}
See \cite{Ezer86} for the basic method.
\begin{options}
\option{steps}{Maximum number of time steps [10000].}
\option{conv}{Convergence criteria for im-propagation [$10^{-5}$ if diag=true, $10^{-8}$ if diag=false].}
\option{Nef}{Number of efs to calculate [20].}
\option{start}{Continue an aborted ef calculation.}
\end{options}

\subsubsection*{Autocorrelation}
\begin{options}
\option{steps}{Number of time steps [10000].}
\option{tol}{Tolerance value for finding peaks in the eigen value specrum [1].}
\option{width}{Width for a sliding window considering the peaks [full specrum].}
\option{read}{Read finished propagation. If \code{read} specifies a directory name then the needed information is obtained form \code{Propagation.meta}.}
\end{options}


\subsubsection*{Examples}
\begin{verbatim}
 eigen {dt=10 steps=10000 ncycle=1 Nef=20 conv=1e-10 dir="ef_test"
   propagator {name=GSPO
         A {name=GridNablaSq mass=1728.4}
         B {name=GridPotential file="pot"}
   }
    wf { file="guess"}
}
\end{verbatim}
Or with the Chebychev propagator:
\begin{verbatim}
eigen {dt=10 steps=10000 ncycle=1 Nef=20 conv="1e-10" dir="ef_test"
   propagator {name=Cheby
         hamiltonian {name=Sum
            A {name=GridNablaSq mass=1728.4}
            B {name=GridPotential file="pot"}
         }
   }
    wf { file="guess" }
}
\end{verbatim}


\section{Filters}
The filter programm is used to analyze a propagation produced by the \verb|propa|.
\begin{options}
\option{path}{Path to propagated fieles.}
\option{dir}{Output Directory.}
\option{fname}{File name for table [\code{expec.dat}]}
\end{options}


\subsubsection*{Examples}
Analyze a 2D propagation postpone:
\begin{verbatim}
qdng {
      filter { path="propa_0.2" dir="analyze_0.2"
           filterdef {
            expeconly { name=GridPosition dim=0 key=x0 header="<x0>"}
            expeconly { name=GridPosition dim=1 key=x1 header="<x1>"}
            expeconly { name=Product header="<x0^2>"
                        A { ref=x0 }
                        B { ref=x0 }
                      }
            expeconly { name=Product header="<x1^2>"
                        A { ref=x1 }
                        B { ref=x1 }
                      }
            expeconly { name=GridMomentum dim=0 key=p0 header="<p0>"}
            expeconly { name=GridMomentum dim=1 key=p1 header="<p1>"}
            expeconly { name=Product header="<p0^2>"
                        A { ref=p0 }
                        B { ref=p0 }
                      }
            expeconly { name=Product header="<p1^2>"
                        A { ref=p1 }
                        B { ref=p1 }
                      }
                   }
          }
} 
\end{verbatim}

Cut out one half of the grid and analyze the results:
\begin{verbatim}
qdng {                                                                                                                                       
   opdefs {                                                                                                                                  
      opdef { name=GridGMat dims=2 gmat="gm" coup=true key=T }                                      
      opdef { name=GridPotential file="pot-clip-0.2" key=V }                                        
      opdef { name=Gobbler dims=2 rp0="0" order=50 residue=true key=GR }                                                                     
      opdef { name=GridPosition dim=0 key=x0 }                                                                                               
      opdef { name=GridPosition dim=1 key=x1 }                                                                                               
      opdef { name=GridMomentum dim=0 key=p0 }                                                                                               
      opdef { name=GridMomentum dim=1 key=p1 }                                                                                               
   }                                                                                                                                         
   filter { path="propa_0.7" dir="0.7" fname="expec_right"                                         
      filterdef {                                                                                                                  
	    expec { ref=GR header="fraction" }                                                                              
	    normalize {}                                                                                                    
	    expeconly { ref=x0 header="<x0>"}                                                                               
	    expeconly { ref=x1 header="<x1>"}                                                                               
	    expeconly { ref=p0 header="<p0>"}                                                                               
	    expeconly { ref=p1 header="<p1>"}                                                                               
	    expeconly { ref=T header=T}                                                                                     
	    expeconly { ref=V header=V}                                                                                     
      }                                                                                                                  
   }
}
\end{verbatim}


\subsection{Defining Filter Lists}
\label{sec:Filters}
Filter lists are evaluated from top to bottom.

\subsubsection*{Sections}
Modes of operation:
\begin{options}
\option{expec}{Apply operator and show expectation value.}
\option{apply}{Just apply the operator.}
\option{expeconly}{Only show the expectation value. Don't modify the wave function.}
\option{normalize}{Normalize the wave function. No further options needed.}
\option{rho}{Extract electronic populations and coherences. Wave function needs to be of Multistate.}
\end{options}

Every filter can also take some options:
\begin{options}
\option{header}{Text to appear in coloumn header [operator name].}
\option{value}{Represenation of the expectation value to plot (real$|$imag$|$complex$|$abs) [real] .}
\option{int}{Accumulate values [false].}
\option{ms}{Specifiy which el. state to act on. Mandatory if \code{rho} is selected as a 
mode of operation. Takes exactly two values which specify the bra and the ket state (\code{ms=i,j}).}
\end{options}


\section{Correlation functions}
\begin{options}
\option{dt}{time step for propagator [required]}
\option{steps}{number of time steps [required]}
\option{stepsint}{ maximum number of intermediate time steps (optional)}
\option{wcycle}{write cycle. The time step in C(t) will be dt*wcycle [1]}
\option{read}{read the baseline propagation (first U) from directory given by 'read' instead (optional)}
\option{fname}{file name for correlation function}
\end{options}

Necessary sections in the input:
\begin{itemize}
 \option{propagator}{}
 \option{ket}{The sequence of operators on the ket side}
 \option{bra}{The sequence of operators on the bra side}
 \option{target}{A target operators if \code{ttype=operator}.}
\end{itemize}

\subsubsection*{Examples}
Correlation function for a transient absorption experiment.
Delivers:
\begin{equation*}
\langle \Psi_1\rvert U^\dagger(t,t_0)\mu U(t,t_1)\mu^\dagger U(t_1,t_0)\lvert\Psi_1\rangle
\end{equation*}
\begin{verbatim}
cfunc {dt="5" steps=600 stepsint=300 wcycle="4" dir="Ctt"
      propagator {name="Arnoldi"
         hamiltonian {name="Multistate"
            m0.0 { name="Sum"
               T { name=GridNablaSq mass=3650 key=T }
               V { name="GridPotential" file="pot_S0" }
            }
            m1.1 { name="Sum"
               T { name=GridNablaSq mass=3650 key=T }
               V { name="GridPotential" file="pot_S0" }
            }
         }
      }
      ket {
        U {}
        probe { name=Multistate unity=false
                m0.1 { name=GridPotential file=mu }
             }
        U {}
      }
      wf { name=Multistate states=2
          wf1 { file="efs-S0/ef_0" }
      }
   }

\end{verbatim}

Same but with two different $\mu$.
Delivers:
\begin{equation*}
\langle \Psi_1\rvert U^\dagger(t,t_0)\mu_2 U(t,t_1)\mu_1^\dagger U(t_1,t_0)\lvert\Psi_1\rangle
\end{equation*}
\begin{verbatim}
cfunc {dt="5" steps=600 stepsint=300 wcycle="4" dir="Ctt"
      propagator {name="Arnoldi"
         hamiltonian {name="Multistate"
            m0.0 { name="Sum"
               T { name=GridNablaSq mass=3650 key=T }
               V { name="GridPotential" file="pot_S0" }
            }
            m1.1 { name="Sum"
               T { name=GridNablaSq mass=3650 key=T }
               V { name="GridPotential" file="pot_S0" }
            }
         }
      }
      ket {
        U {}
        probe { name=Multistate unity=false
                m0.1 { name=GridPotential file=mu1 }
             }
        U {}
      }
      bra {
        U {}
        probe { name=Multistate unity=false
                m0.1 { name=GridPotential file=mu2 }
             }      
      wf { name=Multistate states=2
          wf1 { file="efs-S0/ef_0" }
      }
   }

\end{verbatim}

\section{Optimal Control Theory}
The OCT programm supports Multi Target OCT with various methods:
\begin{itemize}
 \item The Krotov algorithm like described in \cite{Palao03}.
 \item The similar alogrithm with a zero refence field from \cite{Zhu98}.
 \item A feedback driven algorithm designed for operator targets \cite{Zhu98FB}.
 \item Krotov with restrictions in the frequency domain \cite{Gollub08,Gollub10}.
\end{itemize}

\begin{options}
\option{dt}{time step}
\option{steps}{number of time steps}
\option{dir}{Output directory}
\option{method}{OCT method to use [krotov$|$rabitz$|$freq$|$rabitzfb] (default krotov)}
\option{coup}{The type of coupling to optimize [dipole]}
\option{ttype}{target type [operator,overlap]}
\option{phase}{phase sensitive objective. false makes only sense with overlapp target!!! [true|false] (default false)}
\option{ntargets}{Number of targets}
\option{alpha}{The Field restriction value}
\option{iterations}{Maximum number of iterations (default 50)}
\option{conv}{Target convergence. Stop iterating if below convergence (default 10-5)}
\option{shape}{Name of the temporal shapefile}
\option{writel}{Write laser in every iteration to file (default false)}
\option{yfile}{Write the yield table to seperate file (default false)}
\end{options}
It is possible have more than one laser field in the input. Usually the OCT programm
is looking for laserfields in the Hamiltonian. If you want some of them to be fixed e.g. not in
the included in the optimization labeling has to be used.
In that case the active fields have to marked with an attribute \verb|label=OptLaser|.

Necessary sections in the input:
\begin{itemize}
 \option{propagator}{}
 \option{wfi<index>}{One or more initial wave functions.}
 \option{wft<index>}{The respective number of target wave functions if \code{ttype=overlap}.}
 \option{target}{A target operators if \code{ttype=operator}.}
\end{itemize}



\subsubsection*{Examples}
Simple IR transition from $v=0$ to $v=1$:
\begin{verbatim}
oct {alpha="1" dt="10" steps="3000" shape="shape" iterations="30" writel="true"
     method="krotov"
        propagator {name="GSPO"
               A {name="GridNablaSq" mass="1715.65"}
               B {name="GridPotential" file="potX"}
               C {name="GridDipole" file="dipmom" laser="guess_laser"}
        }
        wfi0 {name="GridCartesian" file="ef_0" normalize="true"}
        wft0 {name="GridCartesian" file="ef_1" normalize="true"}
} 
\end{verbatim}


\chapter{Operator modules}
\section{Propagators}

\begin{center}
 \begin{tabular}{lll}
 \hline\hline
  Propagator & time step & non-hermitian $\hat H$ \\\hline
  Split operator (GSPO) & short & yes \\
  Cheby & long &  yes \\
  Short Iterative Lanczos SIL & short & no \\
  Arnoldi & short & yes \\
  RK4 & very short & yes \\\hline\hline
 \end{tabular}
\end{center}

\subsection{GSPO}
General split operator scheme:
\begin{equation}
\label{eq:defSPO}
\hat U = e^{-i(\hat A+ \hat B+..)dt} = e^{-\frac{i}{2} .. dt} e^{-\frac{i}{2} \hat B dt} e^{-i \hat A dt} e^{-\frac{i}{2} \hat B dt} e^{-\frac{i}{2} .. dt}
\end{equation}
The Propagator takes operators which has a diagonal represenation available.
The sections are named according to the operator names in \ref{eq:defSPO}. Where $A$ is always the inner operator.
If operators $C$ and higher are used they have to commute with $B$ to valid.
The corresponding Hamiltonian is: $\hat H = A + B + ...$.
% Multiple electronic states are possible via ODMultistate

\subsubsection*{Examples}
Propagation on a one dimensional grid.
\begin{verbatim}
propa {dt="4" steps="4"
    propagator {name="GSPO"
       A {name="GridNablaSq" mass="1763.2"}
       B {name="GridPotential" file="potX"}
    }
    wf0 {name="GridCartesian" file="ef_0"}
}
\end{verbatim}
% Two electronic states with diabatic coupling. \textbf{Example is may be broken.}
% \begin{verbatim}
% <propa dt="4" steps="4">
%         <propagator name="GSPO">
%         <A name="DMultistate">
%                 <m0.0 name="GridNablaSq" dims="1" mass0="1763.2"/>
%                 <m1.1 name="GridNablaSq" dims="1" mass0="1763.2"/>
%         </A>
%         <B name="DMultistate">
%                 <m0.1 name="GridPotential" file="nacmeCD"/>
%                 <m1.0 name="GridPotential" file="nacmeCD"/>
%         </B>
%         <C name="DMultistate">
%                 <m0.0 name="GridPotential" file="potX"/>
%                 <m1.1 name="GridPotential" file="potA"/>
%         </C>
%         </propagator>
%         <wf name="Multistate" states="2" normalize="true">
%                 <wf0 name="GridCartesian" file="ef_4"/>
%         </wf>
% </propa> 
% \end{verbatim}

\subsection{Cheby}
The Chebychev propagator \cite{Ezer84} needs a subsection \verb|hamiltonian|.
Suitable and very efficient for long time steps. Non-hermitian Hamiltonians are supported.
\begin{options}
 \option{order}{desired max order of recursion, if left empty it's chosen automatically.}
 \option{scaling}{The scaling of the Hamiltonian. Can only be used in combination with order.}
 \option{prec}{The lower limit precission for the series convergence on the norm: $prec < 1-\langle psi|psi \rangle$}
\end{options}

\subsubsection*{Example}
\begin{verbatim}
<propagator name="Cheby">
  <hamiltonian name="Sum">
    <Tkin name="GridNablaSq" dims="1" mass="1763.2"/>
    <Vpot name="GridPotential" file="potX"/>
  </hamiltonian>
</propagator>
\end{verbatim}

\subsection{Short Iterative Lanczos}
The SIL propagator \cite{needed} needs a subsection \verb|hamiltonian|.
Suitable for short time steps or combination with time dependent Hamiltonians.
Non-hermitian Hamiltonians are not supported.
\begin{options}
 \option{order}{Maximum Lanczos basis size [20].}
 \option{err}{Approximated series error. Series Expansions stops when value is reached [$10^{-12}$]}
\end{options}

\subsubsection*{Example}
\begin{verbatim}
propagator { name=SIL order=32
   hamiltonian {name=Sum
      Tkin {name=GridNablaSq mass=1763.2}
      Vpot {name=GridPotential file=potX}
   }
}
\end{verbatim}

\subsection{Arnoldi}
Like SIL but supports non-hermitian Hamiltonians

\begin{options}
 \option{order}{Maximum Lanczos basis size [20].}
 \option{conv}{Approximated series error. Series Expansions stops when value is reached [$10^{-10}$]}
\end{options}

\subsection{RK4}
This is for testing purposes only. Impletments a fourth order Runge-Kutta propagation scheme.
Can be us with non-hermitian and time dependent Hamiltonians. Note no intermediate time steps
are implementet. $\hat H(t)$ is only evaluated at $t$.
The RK4 propagator \cite{needed} needs a subsection \verb|hamiltonian|.

\section{General Operators}

\subsection{Sum}
Represents a sum of other operators.

\subsection{Product}
Represents a product of other operators.
Operators are applied in reverse order: Last operator in the list is applied first.

\subsection{Normalize}
Helper which normalizes a wave function when applied.

\subsection{Scalar}
Helper which applies a constant to a wave function.
\begin{options}
 \option{value}{The constant to apply [1.0]}
\end{options}

\subsection{Projection}
Projection operator consisting of a set of given wave functions. The set is read from
a series of existing wf files.
\begin{equation}
 \hat P = \sum_i |\phi_i\rangle \langle \phi_i|
\end{equation}
or
\begin{equation}
 \label{eq:InvProj}
 \hat Q = 1 - \hat P
\end{equation}
\begin{options}
\option{positive}{Set the sign of the projector (false=negative) [\code{true}].}
\option{files}{Read wavefunctions set $\{\phi_i\}$ from a sequence of files. give base name here.}
\option{num}{Number of files to read [1].}
\option{start}{Start with index [0].}
\option{step}{Step size trough the sequence [1].}
\option{inv}{Apply the inverse projection operator 1-P like in eq. \ref{eq:InvProj} [false]}
\end{options}

\section{Multiple electronic states}
It's possible to combine several electronic states as tensor product. The corresponding multistate
modules allows for defining the matrix elements and states in freely.
If an operator is defined as multistate the wave function has to be defined in the same way e.g. same
number of states (see sec. \ref{sec:WFMultistate}).

The name of the operator is simply \verb|Multistate|. The matrix elements are given in the following way:
\begin{itemize}
 \item The indices are encoded in the section name by \verb|m<row>.<col>|.
 \item The number of states is automatically the maximum index number of the given diagonals.
 \item Off diagonals are only needed to be given once. They are by default assumed to be the adjoint ($H_{ij} = H_{ji}^\dagger$)
 \item Every matrix element can be represented by an arbitrary operator.
 \item Empty off-diagonals are zero.
\end{itemize}

\begin{options}
 \option{states}{Number of states. Usually chosen automatically. Only needed if the diagonal is sparse.}
 \option{unity}{Defines howto handle empty diagonal elements. \code{true} treats empty element as 1 [\code{false}].}
 \option{nonhermitian}{Turn off assumption that $H_{ij} = H_{ji}^\dagger$ (if \code{true}) [\code{false}].}
\end{options}


\subsubsection*{Example}
Two electronic states coupled a laser field.
\begin{verbatim}
 <hamiltonian name="Multistate">
   <m0.0 name="Sum">
       <Tkin name="GridNablaSq" mass="1763.2"/>
       <Vpot name="GridPotential" file="potX.bin"/>
   </m0.0>
   <m1.1 name="Sum">
       <Tkin name="GridNablaSq" mass="1763.2"/>
       <Vpot name="GridPotential" file="potA.bin"/>
   </m1.1>
   <m1.0 name="GridDipole" file="dipXA.bin" laser="Laser290nm_4fs_pp"/>
</hamiltonian>
\end{verbatim}


\section{Grid Based Operators}

\subsection{Kinetic Energy in Cartesian Coordiantes (GridNablaSq)}
Kinetic energy operator for cartesian or cartesian-like coordinates.
\begin{equation}
 \hat T = -\sum_i \frac{1}{2m_i}\frac{\partial^2}{\partial x_i}
\end{equation}

\begin{options}
 \option{dim}{Number of dimensions [\# given in mass]}
 \option{mass}{Define masses for $m_i$. If mass is $= 0$ means not to apply the operator in that dimension}
 \option{method}{\code{FFT} or \code{HOFD}.}
 \option{order}{Finite differencing (FD) order (max. 50) [8].}
 \option{pbc}{FD periodic boundary conditions [false].}
\end{options}

\subsubsection*{Example}
\begin{verbatim}
 Tkin { name="GridNablaSq" mass="1728.4" }
\end{verbatim}

\subsection{Perfect Matched Layer (GridPML)}
Use a cartesian type kinetic energy operator with a perfect matched layer (PML) \cite{Nissen10} as
absorbing boundary conditions.

\begin{options}
 \option{dim}{Number of dimensions [\# given in mass]}
 \option{mass}{Define masses for $m_i$. If mass is $= 0$ means not to apply the operator in that dimension}
 \option{method}{\code{FFT} or \code{HOFD}.}
 \option{order}{Finite differencing (FD) order (max. 50) [8].}
 \option{pbc}{FD periodic boundary conditions [false].}
 \option{lmap}{PML map: Where to put the PMLs. Give a number for each dimension: 0 = none, 1=begin of grid, 2=end grid, 3=both ends.}
 \option{gamma}{}
 \option{thick}{The thickness in number of grid points [10]. One number per dimensions.}
 \option{k}{Optimal momentum. One number per dimensions.}
 \option{p}{Mononomial order. One number per dimensions.}
 \option{smax}{$\sigma_{max}$. Chosen automatically by default.}
 \option{otol}{Approximated error tolerance at the edge the grid. One number per dimensions.}
\end{options}

\subsection{Kinetic Energy in the G-Matrix Represenation (GridGmat)}
Kintec Hamiltonian in G-matrix represenation \cite{Stare03}. With kinetic 
couplings:
\begin{equation}
 T_{kin} = -\frac{\hbar^2}{2} \sum_r^N \sum_s^N \frac{\partial}{\partial q_r} G_{rs}(q) \frac{\partial}{\partial q_s}
\end{equation}
without kinetic coupling:
\begin{equation}
 T_{kin} = -\frac{\hbar^2}{2} \sum_r^N \frac{\partial}{\partial q_r} G_{rr}(q) \frac{\partial}{\partial q_r}
\end{equation}
The Matrix elements can either be given as position dependent elements or they can be constant.
In the first case they are read from a grid file. The naming scheme is like:
\begin{itemize}
 \item \verb|basename_00.op| ($+$ \verb|.meta|)
 \item \verb|basename_10.op| ($+$ \verb|.meta|)
 \item \verb|basename_11.op| ($+$ \verb|.meta|)
\end{itemize}
The base name for the fiels is defined via \verb|gmat|. Constant elements are given by the \verb|G| option.
Mixed definitions of position dependent and constant elements are possible.

\begin{options}
\option{dims}{Number of internal coordiantes.} 
\option{gmat}{G-matrix element base file name.}
\option{coup}{Turn on/off kinetic coupling [true].}
\option{G}{Constant G-matrix elements (given as lower triangular lin. matrix).}
\option{Gdef}{Define source of matrix elements: c = constant, g = grided, x=none (=0) (given as lower triangular lin. matrix)}
\end{options}

\subsubsection*{Example}
\begin{verbatim}
 Tkin { name="GridGMat" dims="2" gmat="g" }
\end{verbatim}

\subsection{Potentials (GridPotential)}
Represents Potentials and other simple grided operators:
\begin{equation}
 V(\vec{x}) = \hat O(\vec{x}) \Psi(\vec{x})
\end{equation}
The only option needed in the input is \verb|file| to specify the file name (w/o extension).
The other options are defined in the meta data.
\begin{options}
 \option{dims}{Number of dims.}
 \option{n<dim>}{Number of grid points in dim.}
 \option{min<dim>}{$x_{min}$ in dim.}
 \option{max<dim>}{$x_{max}$ in dim.}
\end{options}

The following modifier options are recognized in the input file:
\begin{options}
 \option{setzero}{Set the minimum to zero. [false]}
 \option{offset}{Define a constatnt offset value which is added to the potential. [0]}
 \option{scale}{Scale factor which is multiplied to potential[1]}
\end{options}
The operations are carried out in the order given above.

\subsubsection*{Example}
\begin{verbatim}
<Vpot name="GridPotential" file="pot"/>
\end{verbatim}
Reads the binary grided data set from \verb|pot.op| and the meta data from \verb|pot.meta| which may contain for example:
\begin{verbatim}
dims=2
xmin0=0.9331
xmax0=2.329
xmin1=-8.2885
xmax1=9.3468
N1 = 256
N0 = 128
\end{verbatim}

\subsection{Dipole Operator for Laser Interaction (GridDipole)}
Impletments an interaction between a dipole and laser field.
\begin{equation}
 V_c(\vec{x},t) = -\mu(\vec{x}) E(t)
\end{equation}
The dipole moment has the same format as a potential. The laser
can be either generated or read in from a binary file.
This operator can be used for infrared exciatations as as for
the laser coupling between two electronic states.

\begin{options}
 \option{file}{Read from fiven binary file.}
 \option{gen}{Indicates that laser should be generated. See sec. \ref{sec:LaserFields} for details.}
\end{options}


\subsubsection*{Example}
Two electronic states
\begin{equation}
 \hat H = \begin{pmatrix}T + V & -\mu E(t) \\ -\mu E(t) & T + V\\\end{pmatrix}
\end{equation}
\begin{verbatim}
<hamiltonian name="Multistate">
  <m1.0 name="GridDipole" file="dipXA.bin" laser="Laser290nm_4fs_pp"/>
  <m0.0 name="Sum">
    <Tkin name="GridNablaSq" mass="1763.2"/>
    <Vpot name="GridPotential" file="potX.bin"/>
  </m0.0>
  <m1.1 name="Sum">
    <Tkin name="GridNablaSq" mass="1763.2"/>
    <Vpot name="GridPotential" file="potA.bin"/>
  </m1.1>
</hamiltonian>
\end{verbatim}
Simple IR interaction
\begin{equation}
 \hat H = \hat T + \hat V  -\mu E(t)
\end{equation}
\begin{verbatim}
<hamiltonian name="Sum">
  <Tkin name="GridNablaSq" mass="1763.2"/>
  <Vpot name="GridPotential" file="potX"/>
  <Vcoup name="GridDipole" file="dipX" laser="Laser290nm_4fs_pp"/>
</hamiltonian>
\end{verbatim}

\subsection{Two-Photon Raman Laser Interaction (GridRaman)}

\subsection{Flux Operator}
Propabilty flux through a surface/hyper surface. The formulation
is kept general and not bound to cartesian coordiantes only.
\begin{equation}
 \int_S \vec{j} dV = -\frac{2}{\hbar} \int_V \Im(\Psi^* \hat T \Psi)dV
\end{equation}
The operator needs the kinetic operator $T$ and an operator $V$
defining the enclosing volume element of the surface $S$.
Give the operators in exactly this order.

\subsubsection*{Example}
Use it for example in the filter section to collect the outgoing propability:
\begin{verbatim}
filterpost {
  expeconly {
     name=Flux int=true
     T { ref=T }
     G { name=Gobbler dims=1 lp0=5 order=100 }
  }
}
\end{verbatim}
Collect the outgoing flux of a specific electronic state:
\begin{verbatim}
filterpost {
  expeconly { name=Flux int=true
    T { name=Multistate states=8 unity=false
       m3.3 { ref = T}
      }
    G { name=Multistate states=8 unity=false
      m3.3 { name=Gobbler dims=1 lp0=5 order=100 }
      }
  }
}
\end{verbatim}

\subsection{Nabla Operator (GridNabla)}
Nabla operator:
\begin{equation}
 \nabla  \equiv \sum_n \frac{\partial}{\partial x_n}
\end{equation}

\begin{options}
 \option{dim}{If the operator acts only on the given dimension. Default is to sum up over all.}
 \option{factor}{Prefactor [1]}
\end{options}

\subsection{Momentum Operator (GridMomentum)}
Momentum operator:
\begin{equation}
 \hat p = \sum_n -i \frac{\partial}{\partial x_n}
\end{equation}

\begin{options}
 \option{dim}{If the operator acts only on the given dimension. Default is to sum up over all.}
  \option{factor}{Prefactor [1]}
\end{options}

\subsection{Position Operator}
\begin{equation}
 \hat \vec{x} = \sum_i \hat x_i
\end{equation}

\begin{options}
 \option{dim}{If the operator acts only on the given dimension. Default is to sum up over all.}
  \option{scale}{Prefactor [1]}
\end{options}

\subsubsection*{Example}
\begin{verbatim}
filterpost {
  expeconly { name=Multistate states=2 unity=false 
    m0.0 { name=GridPosition }
  }
}
\end{verbatim}
Be careful, a single electronic is not normalized to 1 here.

\subsection{Momentum kick (GridKick)}
Add momentum $k$ to a wave function
\begin{equation}
 e^{i s \vec{k} \vec{x}} \Psi
\end{equation}
\begin{options}
 \option{k}{Give momentum vector $k$ as a list.}
 \option{scale}{Scale the momentum vector with $s$ [1.0].}
\end{options}

\subsubsection*{Example}
Modifiy the inital condition in 2D propagation:
\begin{verbatim} 
filterinit {
            apply { name=GridKick k="-60.8,10" }
           }
\end{verbatim}


\subsection{Non-adiabtic couplings}
Non-adiabtic couplings between electronic states. Can handle conical intersections.
For the simple 1D case:
\begin{equation}
 K \Psi = \frac{1}{m}
\begin{pmatrix}
 0 & 1 \\
-1 & 0 \\
\end{pmatrix}
\left (
f_{12} \partial_R \Psi + \dfrac{1}{2} \Psi \partial_R f_{12}
\right )
\end{equation}
For the more general ND-case:
\begin{equation}
 K \Psi =
\begin{pmatrix}
 0 & 1 \\
-1 & 0 \\
\end{pmatrix}
\sum_j \left (
\tilde f_{12}^{(j)} \partial_{q_j} \Psi + \dfrac{1}{2} \Psi \partial_{q_j} \tilde f_{12}^{(j)}
\right )
\end{equation}
In the ND-case the mass must be included in the in data of the coupling file.

\begin{options}
 \option{dims}{Number of dimensions.}
 \option{mass}{Reduced mass (only for 1D).}
 \option{sign}{Sign of the matrix element either 1 or -1 [1].}
 \option{file}{Coupling element $f_{12}$.}
\end{options}

\subsubsection*{Example}
\begin{verbatim}
Coupling { name=Multistate
  m1.0 { name=GridNAC file="nacmeT" dims=1 sign=-1 mass=1763.2 }
  m0.1 { name=GridNAC file="nacmeT" dims=1 mass=1763.2 }
}
\end{verbatim}


\subsection{Gobbler}
Masking function (Butterworth) which can also be used as negative imaginary potential.

\begin{options}
\option{rp<dim>}{a "right pass" leaving everything on the right hand side (larger x values). dim is the number of the dimesion the gobbler is applied to (starting with dimension 0).}
\option{lp<dim>}{a "left pass" leaving everything on the left hand side (smaller x values). dim is the number of the dimesion the gobbler is applied to (starting with dimension 0).}
\option{order}{filter order. determines the cutoff sharpness.}
\option{nip}{If present use the gobbler as negative imaginary potential.}
\option{gain}{Supply a gain value. 1 per default. Only makes sense in combination with nip.}
\option{residue}{Don't calculate a expectation value but the residue after application. Calculates $\langle \Psi|G^\dagger G|\Psi\rangle$ instead of $\langle \Psi|G|\Psi\rangle$.}
\end{options}

\subsubsection*{Example}
\begin{verbatim}
   <filterpost>
        <expec name="Gobbler" rp1="-8.1" lp1="+9.2" order="50"/>
   </filterpost>
\end{verbatim}


\section{Matrix Represenation}
\subsection{Hermitian Matrix}
Represents a real symetric Matrix.
Storage in binary file is in full format.

\subsubsection*{Example}
\begin{verbatim}
propa {dt="1" steps="8000" dir="propa2" wcycle="1000" nfile="norm"
       spectrum="autocorr" proj0="true"
   propagator {name="Cheby"
       hamiltonian {name="HermitianMatrix" file="Hmat"}
   }
   wf {name="Level" file="wfvec"}
} 
\end{verbatim}


\subsection{Hermitian Matrix Laser Coupling}
\begin{equation}
 V_c(t) = -\mu E(t)
\end{equation}

\begin{options}
 \option{file}{Read from fiven binary file.}
 \option{gen}{Indicates that laser should be generated. See sec. \ref{sec:LaserFields} for details.}
\end{options}

\subsubsection*{Example}
\begin{verbatim}
propa { dt="0.5" steps="629" wcycle="4096" dir="TLPL-GSPO" spectrum="spectrum.dat" proj0="true"
        propagator { name="GSPO"
          A { name="HermitianMatrix" file="LevelH2mat"}
          B { name="MatDipole" file="LevelDIPmat" laser="LevelLaser_w0.15"}
        }
        wf { name="Level" file="Level_ef_0"}
        filterpost {
          expeconly { name="Projection" files="Level_ef" start="0" }
          expeconly { name="Projection" files="Level_ef" start="1" }
        }
} 
\end{verbatim}

\section{Laser Fields}
\label{sec:LaserFields}
\subsection{On the fly generation.}
Operators which include laser fields can generate the field on the fly if the option gen is used.
\begin{equation}
E(t) = E_{max} e^{-(t-t_0)^2/2\sigma^2} \cos(\omega_0 (t-t_0)+ b(t-t_0)^2+\phi) 
\end{equation}
A single Gaussian pulse is generated with \verb|gen=gauss|. A double pulse can be created with
\verb|gen=gauss2|.
\begin{options}
 \option{Emax}{Maximum [$0.0001$].}
 \option{t0}{Center of envelope [$T/2$].}
 \option{sigma}{Width [$T/8$].}
 \option{w0}{Center frequency [$0.05695 = 800$ nm]}
 \option{phi}{CEP [0].}
 \option{b}{Chirp parameter [0].}
 \option{lasero}{File name for dumping the generated field.}
\end{options}
In the case of double pulse, the parameters of the second pulse can controlled by adding a 2 to
name of the option (Emax2, t02, sigma2, w02, phi2, b2).

\subsubsection*{MATLAB code snippet for visualization}
\begin{verbatim}
t = 0:1000;
dt = 1;

t=t*dt;

Emax = 0.0001;
t0 = 500;
w0 = 0.05695;
phi = 0;
b = 0.0;
sigma=500/8;

plot(t, Emax * exp(-(t-t0).^2/2/sigma^2) .* cos(w0*(t-t0) + phi + b*(t-t0).^2)); 
\end{verbatim}


\chapter{Wave Functions}
Wave function can be defined in two ways -- either it is just a pure wave function from a
file or it is compostion of wave functions like a super position or a tensor product like
multiple electronic states.

The following options apply to all modules.
\begin{options}
 \option{name}{Name of the wave function module. Needs not to be given if the \code{CLASS} option in the meta data is present or the pbuf file format is used.}
 \option{normalize}{Normalizes the the subsections [false].}
 \option{phase}{Applies a phase factor $e^{i\phi}$ to the subsection (in units of $\pi$).}
 \option{coeff}{The coefficient $c_n$ [1.0]}
 \option{coeff2}{The squared coefficient $c^2_n$ [1.0]}
\end{options}

\section{Wave function from file}
The only mandatory option here is \verb|file| which gives the base name of the wave function file
(without .wf ending).

\section{Linear combination}
The name of the module is \verb|LC|.
\begin{equation}
 \Psi = \sum_n \phi_n
\end{equation}

\subsubsection*{Example}
\begin{verbatim}
<wf name="LC" normalize="true">
 <wf1 file="ef_0" coeff2="0.25"/>
 <wf2 file="ef_1" coeff2="0.5"/>
 <wf3 file="ef_2" coeff2="0.5"/>
</wf>
\end{verbatim}


\section{Multiple electronic states}
\label{sec:WFMultistate}

Multistate wave functions are composed of ``single state'' wave functions like the operator.
The name of the module is simply \verb|Multiple|. The subsections have to follow the naming
scheme \verb|wf<index>|. The option \verb|states| defines the number of states if it can not
be guessed automatically from the number of subsections. An empty state is initalzed with zeros
by default. At lest one wave function must be given.

\subsubsection*{Example}
Define a superposition of two electronic states:
\begin{verbatim}
<wf name="Multistate" states="2" normalize="true">
  <wf0 file="ef_0"/>
  <wf1 file="ef_0"/>
</wf>
\end{verbatim}

\section{Remapping of electronic states}
In the protobuf file format multi state wave functions are written into a single file,
instead of one file per state. With the \verb|Multimap| module single states of an input 
can be picked or reordered.

\begin{options}
 \option{statemap}{Indicies for the new ordering. The position of the index represents the position in the new wave function.}
 \option{coeff}{The coefficient for a particular state [1.0]}
\end{options}

\subsubsection*{Examples}
Pick just single state from a multi state input file (i.e. pick state 2). This is special case, since
the output will not be of type \verb|Multistate| but of the particular type of the wave function.
\begin{verbatim}
wf { name=Multimap statemap=2
  wf { file=WF }
}
\end{verbatim}

Select a  subset of states, change the order, and scale them This does the following mapping:
\begin{align}
\begin{pmatrix}
\Psi_0 \\ \Psi_1 \\ \Psi_2 \\ \Psi_3
\end{pmatrix}
\rightarrow
\begin{pmatrix}
0.2 \Psi_3 \\ 0.3 \Psi_0 \\ 0.4\Psi_1
\end{pmatrix}
\nonumber
\end{align}
\begin{verbatim}
wf { name=Multimap statemap=3,0,1 coeffs=0.2,0.3,0.4
  wf { file=WF }
}
\end{verbatim}

\bibliographystyle{acm}
\bibliography{qdng}

%\thebibliography

\end{document}


